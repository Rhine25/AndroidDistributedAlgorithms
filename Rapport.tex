\documentclass[a4paper,10pt]{report}
\usepackage[utf8]{inputenc}
\usepackage[francais]{babel}
\usepackage{graphicx}
\usepackage{titlesec}
\usepackage{fancyvrb}
\usepackage{verbatim}
\usepackage{pdfpages}
\usepackage{hyperref}
\usepackage{todonotes}

% Title Page
\title{Développement d'un environnement pour algorithmes distribués sur Android}
\author{Célia Dehors, Vincent Durand, Justine Romero}
\date{Année 2017-2018}

\makeatletter
\let\thetitle\@title
\let\theauthor\@author
\let\thedate\@date
\makeatother

\makeatletter
\newskip\@bigflushglue \@bigflushglue = -100pt plus 1fil
\def\bigcenter{\trivlist \bigcentering\item\relax}
\def\bigcentering{\let\\\@centercr\rightskip\@bigflushglue%
\leftskip\@bigflushglue
\parindent\z@\parfillskip\z@skip}
\def\endbigcenter{\endtrivlist}
\makeatother

\titleformat{\chapter}[hang]{\bf\huge}{\thechapter}{2pc}{}

\begin{document}  
  \begin{titlepage}
    \centering
    \vspace*{0.5 cm}
    \includegraphics[scale = 0.05]{logo_univ.png}\\[1.0 cm]   % University Logo
    \LARGE Université de Franche-Comté\\[2.0 cm]   % University Name
    \Large Rapport de Projet Semestriel\\               % Course Code
    \large Master Informatique, 2ème année\\[2.5 cm]               % Course Name
    \rule{\linewidth}{0.2 mm} \\[2.5 mm]
    { \huge \bfseries \thetitle}\\
    \rule{\linewidth}{0.2 mm} \\[2.0 cm]
    
    \begin{minipage}{0.6\textwidth}
        \begin{flushleft} \large
            \emph{Étudiants:}\\
            \theauthor
        \end{flushleft}
    \end{minipage}~
    \begin{minipage}{0.4\textwidth}
        \begin{flushleft} \large
            \emph{Tuteur:}\\
            M. Laurent PHILIPPE
        \end{flushleft}
    \end{minipage}\\[3.0 cm]
    {\large \thedate}\\[2 cm]
    \vfill
    
\end{titlepage}
\chapter*{Remerciements}
%Table des matières
\begin{changemargin}{0cm}{0cm}{-10cm}{0cm}
\tableofcontents
\end{changemargin}

\chapter{Introduction}
  \paragraph{}
  Différents projets ont été proposés aux élèves de deuxième année de Master pour cette année 2018. Dans notre cas nous avons choisi le sujet suivant : "Développement d’un environnement pour les algorithmes distribués sur Android".
  
  Ce projet nous a intéressé car il mélangeait différents domaines appréciés par les membres de notre trinome soit la synchronisation distribuée et le développement Android.
  L'aspect distribué permet aux utilisateurs d'avoir leur application sur leur téléphone qui partage des données avec les autres terminaux mobiles inter-connectés.
  
  Quelles sont les solutions et les méthodes que nous avons appliquées pour réaliser ce projet ?
  
  Ce rapport se compose de cinq grands axes. Nous étudierons tout d'abord le sujet du projet en détail puis nous présenterons les recherches préliminaires au développement.Par la suite nous allons décrire l'architecture du projet avec ses différentes couches d'implémentation et l'algorithme implémenté afin de tester les fonctionnalités crées. Enfin nous reviendrons sur les outils utilisés.
\chapter{Présentation du projet}
  \section{Sujet}
  L'objectif de ce projet est de réaliser un outil permettant d'exécuter des algorithmes distribués sur des téléphones portables. Pour ce faire les appareils doivent pouvoir communiquer de manière synchronisée. Il doit être possible de rédiger simplement des algorithmes afin de les dérouler et d'observer son comportement (visualiser les messages et leur contenu). Ces algorithmes ont vocation à être utilisés pour créer des applications distribuées.
  
  Le projet est à réaliser en java sous android.
  \section{Problématiques}
  Les questions que nous nous sommes posés sont les suivantes : 
  \begin{itemize}
    \item Quelles technologies allons nous utiliser ?
    \item Quelle architecture allons nous mettre en place pour faciliter la rédaction des algorithmes ?
    \item \todo{Trouver une troisième question}
  \end{itemize}
\chapter{Recherche des technologies de communication}
  \paragraph{}
  L'objectif de ce projet est la réalisation d'un environnement d'exécution et de communication permettant le développement et la démonstration d'algorithmes distribués sur des appareils Android. Notre travail consiste donc au développement d'une application Android possédant son propre système de communication et capable de faire exécuter divers algorithmes.
  Notre travail se découpe en plusieurs phases :
  La première est l'étude des différentes technologies de communications disponibles sur les appareils Android
  La deuxième est l'implémentation du système de communication dans la technologie choisie
  La troisième est la mise en place d'une API permettant aux développeurs de mettre en place les algorithmes souhaités, en l'occurence des algorithmes distribués, sur l'application créée.
  \section{Bluetooth}
    \subsection{Historique}
    \subsection{Fonctionnement} \todo{pap}
    \subsection{Avantages}
    \subsection{Inconvénients}
  \section{Wifi Direct ou juste Wifi} \todo{DECIDER}
    \subsection{Historique}
    \subsection{Fonctionnement}
    \subsection{Avantages}
    \subsection{Inconvénients}
  \section{NFC}
    \subsection{Historique}
    \subsection{Fonctionnement}
    \subsection{Avantages}
    \subsection{Inconvénients}
  \section{Choix de la technologie}
  
%   \section{Comparaison Bluetooth/Wifi} \todo{virer cette partie}
%     \paragraph{}
%     Notre première étape est l'étude des différentes technologies disponibles. Notre attention se porte sur deux technologies principales bien connues : Bluetooth et Wifi.
%     Nous recherchons tout d'abord des informations sur le web.
%     Wifi semble être davantage adapté à notre usage. Il est plus rapide, stable, fiable, accepte plus de connexions.
%     Bluetooth est quant à lui plus léger.
%     Puis afin d'avoir une comparaison plus tangible des deux technologies, nous décidons d'implémenter deux applications distinctes utilisant chacune une des technologies. Cela dans le but de comparer autant les performances/limites des technologies que leur facilité de mise en place.
%     -> Conclusion
%     L'implémentation du Bluetooth est bien plus rapide que celle du wifi, la documentation est plus simple et plus accessible. Nous essayons d'implémenter le wifi mais le temps passant, nous préférons partir sur le bluetooth qui fonctionne déjà.
    
\chapter{Développement de l'architecture de l'application}
  \paragraph{}
  Dans cette partie nous allons présenter la façon dont l'application a été conçue et expliquer le fonctionnement de celle-ci. L'application a été architecturée en plusieurs couches pour introduire plusieurs niveaux d'abstraction, séparer les rôles et simplifier le développement.
  
  \section{Couche bas niveau}
    \paragraph{}
    La couche bas niveau de notre application permet la communication entre plusieurs appareils et va servir de support à l'implémentation des algorithmes distribués. Elle est constituée des mécaniques de connexion, de routage et d'échange de paquets sur le réseau afin de permettre une mise en place simple des algorithmes. C'est elle qui a été concue en premier car elle est le soubassement de toutes les Fonctionnalités de notre application.
  
    \subsection{Fonctionnement de base}
    \paragraph{}
    La première pierre de l'édifice est la création d'un service Bluetooth central qui va être le garant du bon fonctionnement des échanges entre appareils. C'est lui qui va permettre d'activer ou désactiver la connexion Bluetooth de l'appareil, qui va décider si l'on se rend visible sur le réseau ou non mais surtout de rechercher des appareils disponibles et se connecter à eux pour ensuite communiquer ensemble.
    
    \subsection{Connexion}
    \paragraph{}
    La connexion en Bluetooth entre plusieurs appareils nécessite la création d'un UUID. Un UUID est un identifiant unique composé de 128 bits. Sa taille est telle que l'on considère qu'il n'est pas possible d'avoir de collision en choisissant un UUID au hasard. Dans notre application, cela va permettre de ne pas tenter de se connecter à un appareil qui serait visible en Bluetooth mais qui n'utiliserait pas l'application, et qui par conséquent n'est pas souhaité dans notre réseau.
    Pour qu'une connexion soit réalisée en Bluetooth il est nécessaire d'avoir un client et un serveur. Le serveur va ouvrir une socket sur laquelle le client va venir initier une connexion à partir de l'adresse MAC du serveur.
    La connexion entre plusieurs appareils est gérée par le service Bluetooth. Chaque appareil peut être indistinctement client ou serveur. Pour cela, les mécanismes de connexions sont divisés en deux threads : un thread ayant le rôle de serveur et un thread ayant le rôle de client.
    Lorsque qu'un client et un serveur se sont connectés, un canal RFCOMM est créé permettant la communication des deux appareils, chacun pouvant lire et écrire sur la socket correspondant à la connexion indistinctement du rôle qu'il a joué lors de la phase de connexion.
    
    \subsection{Communication}
    \paragraph{}
    Le mécanisme de communication est implémenté dans un nouveau thread pour permettre le maintien de plusieurs connexions simultanées. Ce thread va constemment attendre qu'un flux de données arrive en entrée sur la socket. Il va bien évidemment aussi être capable d'écrire sur cette socket.
    Tous les messages échangés adoptent la même forme : un expéditeur, un destinataire et le contenu du message constitué lui-même d'un octet correspondant au type de message et d'un tableau d'octets correspondant aux données du message. Le type de message peut être soit un type de message de bas niveau permettant la mise en place des mécanismes de bas niveau de la communication soit un type de message correspondant aux messages de l'algorithme distribué implémenté.
    Lors de la réception d'un message sur la socket, nous récupérons le type de message envoyé et effectuons les traitements adpatés en fonction de celui-ci.
    
    \subsection{Routage}
    \paragraph{}
    La mécanique de bas niveau la plus complexe et sophistiquée est le routage. C'est lui qui va permettre à des appareils qui ne sont pas connectés directement de pouvoir tout de même communiquer entre eux. Cela est nécessaire du fait de la faible portée du Bluetooth afin de pouvoir faire partie du réseau sans pour autant être à portée de tous les autres appareils du réseau.
    
    //table de routage
    //gestion des entrées de la table
    //recherche de chemin vers
    //communication avec un appreil distant
  
  \section{Couche intermédiaire}
    \subsection{Implémentation d'un anneau}
      \subsubsection{Création de l'anneau}
      \subsubsection{Insertion dans un anneau}
      \subsubsection{Fusion des anneaux}
  \section{Couche haut niveau}
    \subsection{API sur le modèle de Visidia}
    \subsection{Algorithmes} \todo{Implémentabilité}
%   \section{Fonctionnalités de base / connexion}
%     \paragraph{}
%     Création d'un UUID afin d'identifier les connexions Bluetooth
%     Ajout des threads pour les rôles de client, de serveur, et de gestion de connexions. \todo{expliquer les Fonctionnalités des threads}
%     Ajout d'une Fonctionnalité pour activer/désactiver le bluetooth depuis l'application sans devoir passer par les paramètres système
%     Refactoring pour déplacer les morceaux de codes dans des petites fonctions afin de rendre le code plus lisible et mieux maintenable
%     Ajout d'un menu proposant les options de se rendre visible sur le réseau, de découvrir les appreils visibles, de se rendre disponible en tant que serveur (avec le UUID) ou de se connecter à un serveur
    
    
%     S'appairer/se connecter à deux puis plus
%     Mode serveur : accepter une connexions
%     Mode client : se connecter à un serveur
%     Mise en place de la connexion (stream entre deux device)
%     L'architecture du service Bluetooth est la suivante : une classe principale BluetoothService composée d'une interface regroupant les types de messages échangés à bas niveau (lecture, écriture, connexion, déconnexion et mécanismes de fonctionnement de la couche basse de l'application) et trois threads permettant l'interconnexion des appareils. Nous avons un thread serveur permettant d'accepter des connexions entrantes, un thread client permettant de se connecter sur un serveur et un thread de connexion établie permettant d'envoyer et de recevoir des messages entre deux appareils connectés.
    
%   \section{Fonctionnalités de communication/ couche basse comm}
%     \paragraph{}
    
%     Envoi de message texte
%     Mise en place routage, sérialisation
%     explication des formats de messages MessagePacket, type et byte[] data
    
%   \section{Couche intermédiaire}
%     \paragraph{}
%     Implémentation des types de messages
%     Implémentation d'un token ring
%       -> réflexion implémentation d'un token ring au niveau api puis implémentation réelle à bas niveau
    
%   \section{Couche haute API algos}
%     \paragraph{}
    
  \subchapter{Développement d'une API permettant de créer des algos distribués}
  \paragraph{}

\chapter{Algorithme distribué implémenté}
\chapter{Outils de développement}
  \paragraph{}
  Utilisation d'Android Studio et versionnement des sources avec un serveur gitlab.
  \section{Android Studio}
  \section{Gitlab}
\chapter{Conclusion}
\end{document}          