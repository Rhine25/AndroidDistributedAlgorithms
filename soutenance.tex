\documentclass{beamer}

\usepackage[francais]{babel}
\usepackage[T1]{fontenc}
\usepackage[utf8]{inputenc}
\usepackage{graphicx}

\usetheme{Warsaw}

\title{Développement d’un environnement pour algorithmes distribués sur Android}
\author{Célia Dehors, Justine Romero, Vincent Durand}
\institute{Université de Franche-Comté}
\date{Année 2017-2018}

\beamertemplatenavigationsymbolsempty
\setbeamertemplate{footline}[frame number]

\logo{\includegraphics[height=20mm]{logo_univ.png}}

\begin{document}
    \begin{frame}[plain]
        \titlepage
    \end{frame}
    \logo{\includegraphics[height=5mm]{logo_univ.png}}
    \begin{frame}
        \tableofcontents
    \end{frame}
    \section{Présentation du projet}
      \subsection{Sujet}
	\begin{frame}
	  \frametitle{Sujet}
	  \includegraphics[height=45mm]{android.jpg}
	  Développement d’un environnement pour algorithmes distribués sur Android.
	  \\Réalisation d’une plateforme qui permet de programmer facilement un algorithme distribué
        \end{frame}
      \subsection{Problématiques}
        \begin{frame}
	  \frametitle{Problématiques}
	  \begin{itemize}
	   \item Quelles technologies pour la communication ?
	   \item Quelle architecture pour l'application ?
	   \item Comment rendre le fonctionnement transparent au développeur ?
	  \end{itemize}

        \end{frame}
    
    \section{Recherche des technologies de communication}
      \begin{frame}
	\frametitle{Les technologies}
	\includegraphics[height=40mm]{technologies.png}
      \end{frame}
      \subsection{Bluetooth}
      \begin{frame}
	  \frametitle{Bluetooth}
	  \begin{block}{Avantages :}
	  \begin{itemize}
	   \item Compatibilité
	   \item Documentation exhaustive
	  \end{itemize}
	  \end{block}
	  \begin{block}{Inconvénients :}
	  \begin{itemize}
	   \item Consommation d'énergie
	   \item Plus lent, moins stable que le Wi-Fi
	  \end{itemize}
	  \end{block}
      \end{frame}
      \subsection{Wi-Fi}
      \begin{frame}
	  \frametitle{Wi-Fi}
	  \begin{block}{Avantages :}
	  \begin{itemize}
	   \item Vitesse
	   \item Compatibilité entre différents appareils
	  \end{itemize}
	  \end{block}
	  \begin{block}{Inconvénients :}
	  \begin{itemize}
	   \item Combatibilité non assurée entre les marques
	   \item Peu de documentation
	  \end{itemize}
	  \end{block}
      \end{frame}
      \subsection{NFC}
      \begin{frame}
	  \frametitle{NFC}
	  \begin{block}{Avantages :}
	  \begin{itemize}
	   \item Communication Full Duplex
	   \item Vitesse
	  \end{itemize}
	  \end{block}
	  \begin{block}{Inconvénients :}
	  \begin{itemize}
	   \item Faible distance requise entre appareils
	  \end{itemize}
	  \end{block}
      \end{frame}
      \subsection{Choix}
      \begin{frame}
	  \frametitle{Choix de la technologie}
	  \begin{center}
	     \includegraphics[height=70mm]{bluetooth.jpg}
      \end{center}
      \end{frame}
    \section{Conception}
    \begin{frame}{Conception}
        \begin{itemize}
            \item Couche bas niveau
            \item Couche intermédiaire
            \item Couche haut niveau
        \end{itemize}
    \end{frame}
      \subsection{Couche bas niveau}
          \begin{frame}
    	    \frametitle{Couche bas niveau}
    	    \begin{itemize}
    	        \item Connexion
    	        \item Routage
    	        \item Communication
    	    \end{itemize}
          \end{frame}
      \subsection{Couche intermédiaire}
      \begin{frame}
	  \frametitle{Couche intermédiaire}
	  \begin{itemize}
	      \item Election
	      \item Anneau
	  \end{itemize}
      \end{frame}
      \subsection{Couche haut niveau}
      \begin{frame}
	  \frametitle{Couche haut niveau}
	  \begin{itemize}
	      \item Liaison entre les couches inférieures et l'algorithme distribué
	      \item Environnement mis à disposition du développeur
	      \item Permet l'envoi et la réception de messages : onMessage() et sendMessage()
	      \item API basée sur le modèle de ViSiDiA
	  \end{itemize}
      \end{frame}
    \section{Algorithme distribué implémenté}
    \begin{frame}{Algorithme distribué implémenté}
    Lamport avec estampilles pour une simulation de demande d’entrée en section critique :
        \begin{itemize}
            \item Demande d'accès en section critique
            \item Gestion des demandes
            \item Libération de section critique
        \end{itemize}
    \end{frame}
    \begin{frame}{Algorithme}
        \includegraphics[scale=0.3]{regle.png}
    \end{frame}
    \section{Bilan}
    \begin{frame}{Bilan}
        Difficultés rencontrées:
        \begin{itemize}
            \item Intégration de l'algorithme
            \item Découverte d'Android
        \end{itemize}
        Améliorations possibles :
        \begin{itemize}
            \item Application multi-plateforme
            \item Affichage du contenu des variables dans l'application
        \end{itemize}
    \end{frame}
\end{document}


%Contexte
    %Cahier des charges
    %Besoin
%Ce qu'on a fait
%Pourquoi comme ça
%Comment au niveau Organisation
%Conclusion
%Bilan
